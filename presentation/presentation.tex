\documentclass{beamer}
\usepackage{graphicx}
\usepackage{ngerman}
\usepackage{haskell}
\usepackage[utf8]{inputenc}
\usetheme{default}
\begin{document}

\begin{frame}{Haskell parallel}
\begin{itemize}
\item Haskell unterscheidet auf Ebene des Typsystems zwischen
  reinen und unreinen Berechnungen (Seiteneffekte)
\item Sehr viel Umstrukturierung durch den Compiler möglich
\item Reine Berechnungen könnten fast ohne weiteres parallelisiert
  werden, insbesondere z.B. $map$
(vgl. Google\textsuperscript{\texttrademark}
  MapReduce\textsuperscript{\texttrademark})
\item Funktionen als First Class Citizens - könnte helfen
\end{itemize}
\begin{definition}{Fazit}
Parallelismus beinahe implizit verfügbar
\end{definition}
\end{frame}

\begin{frame}{Eden}
\begin{itemize}
\item Paralleler Haskell-Dialekt
\item Übersetzt von gewissen Haskell-Funktionen nach z.B. MPI
\end{itemize}
\end{frame}

\begin{frame}{Trans \& Process}

\begin{definition}{Trans}
Typklasse für ``übertragbare'' Daten
\end{definition}

\begin{definition}{Process}
Typklasse für ``übertragbare'' Funktionen
\end{definition}

\end{frame}

\begin{frame}{Spawn}

Spawn u.A. dienen dazu, Berechnungen ``auszulagern'' an einen anderen
Rechner. Vergleichbar mit MPI $spawn$.

Expliziter Parallelismus in Haskell.

\end{frame}

\begin{frame}{Monaden}
\begin{itemize}
\item Haskell-Konstrukt für ``imperative Programmierung''
\item vgl. ``Monoid''
\item Typ höherer Ordnung (``Kind'' $* \rightarrow *$')
\item Beispiel: $print: String -> IO ()$
\item $do$-Syntax: ``imperativer Programmblock''
\end{itemize}

\begin{definition}{IO}
IO a: ``Skript mit Rückgabewert a'' - Achtung, IO ist atypisch!
\end{definition}

Listen als Monade: ``Nicht-Deterministische'' Evaluation ohne
Aufpreis.

\end{frame}

\begin{frame}{ST-Monade}
\begin{itemize}
\item mittels $runST$ umzuwandeln in eine ``reine'' Berechnung:
  $runST :: (forall s. ST s a) -> a$
\item ermöglicht monadische Formulierung von imperativen Programmen
  ohne Abhängigkeit von ``unreinen'' Berechnungen
\item u.A. Arrays, veränderbare Referenzen etc.
\end{itemize}
\end{frame}

\begin{frame}{ST-Monade und Eden}
\begin{itemize}
\item Problem: $Trans (ST x)$ instanziieren?
\item Rein technisch wohl möglich
\item besser erstmal nicht...
\end{itemize}
\end{frame}

\begin{frame}
\begin{itemize}
\item Lieber ``Array'' (reiner Typ) übertragen
\item Noch besser vielleicht als ``Remote Data''
\end{itemize}
\end{frame}

\begin{frame}[fragile]{Aber...}
\tiny
\begin{verbatim}
Som/Parallel.hs:62:12:
    Could not deduce (Trans d, Trans i) arising from a use of `spawn'
    from the context (Show d,
                      Show i,
                      Ix i,
                      DataPoint d,
                      Coordinate i,
                      Inf d,
                      Trans (Array i d))
      bound by the type signature for
                 learndataparallel :: (Show d, Show i, Ix i, DataPoint d,
                                       Coordinate i, Inf d, Trans (Array i d)) =>
                                      Float -> [Array i d] -> d -> [(i, d)]
      at Som/Parallel.hs:(61,1)-(64,9)
    Possible fix:
      add (Trans d, Trans i) to the context of
        the type signature for
          learndataparallel :: (Show d, Show i, Ix i, DataPoint d,
                                Coordinate i, Inf d, Trans (Array i d)) =>
                               Float -> [Array i d] -> d -> [(i, d)]
    In the expression:
      spawn
        (repeat
         $ process
           $ \ s
               -> runST
                  $ do { som_ <- freeze s;
                         findclosest som_ datapoint })
\end{verbatim}
\end{frame}

\begin{frame}
  Haskell-Typisches Problem: Zusammentreffen von
  \begin{itemize}
    \item Mehreren Constraints
    \item Typen höherer Ordnung
    \item Unspezifizierte Typen
  \end{itemize}

  Verlass Dich nie auf Hindley-Milner!
\end{frame}

\begin{frame}[fragile]
  Lösung:
  
  In mehrere Unterfunktionen mit klaren Typen aufgliedern
  
  Hilfreiche Funktion:
  
  \begin{haskell*}
    thawST &::& (Ix i, IArray a e) => a i e -> ST s (STArray s i e)\\
    thawST &= &thaw
  \end{haskell*}

  (Typ eingeschränkt ggü. thaw!)

  2 Zeilen klären eine ganze Menge Typ-Unklarheiten

\end{frame}

\begin{frame}{Remote Data}
\begin{itemize}
\item Verwaltung von Daten als ``Handle''
\item Vermeidet unnötige Übertragung
\item 
\end{itemize}
\end{frame}

\end{document}
